\documentclass{article}
\usepackage{placeins}
\usepackage{graphicx}
\usepackage{subcaption}
\usepackage{listings}
\usepackage{hyperref}
\usepackage{cleveref}
\usepackage{booktabs, siunitx}
\usepackage{geometry}
\usepackage{minted}
\usepackage{subfiles}
\usepackage{indentfirst}
\usepackage[backend=biber]{biblatex}
\usepackage[svgnames,table]{xcolor}

\addbibresource{ref.bib}
\usemintedstyle{emacs}
\geometry{
 a4paper,
 total={170mm,257mm},
 left=20mm,
 top=20mm,
 }
\graphicspath{ {./images/} }

\title{Assignment 2 Report}
\author{Tanat Tangun 630610737}
\date{September 2022}

\begin{document}
\maketitle
This report is about the result of my fuzzy logic implementation on Rust language for 261456 - INTRO COMP INTEL FOR CPE class
assignment. This report will be about how fuzzy-logic can help in trading securities and other assets. 
If you are interested to know how I implement the fuzzy-logic and use it to help in trading securities and other assets
, you can see the source code on my 
\href{https://github.com/RiwEZ/MLPOnRust}{Github repository} or in this document appendix.

\section*{Introduction}
The term ``securities and other assets" in our report include securities such as stock, bonds, currencies, and other assets
i.e. crypto-currency, house price, etc. In this report, we will focus on crypto-currency such as Bitcoin and Ethereum. Before we delve 
further, let's understand that we are not trying to create more profit than any other techniques but we are trying to learn how
to create fuzzy logic and explore how it might be useful for trading.

\subsection*{Technical Indicators}
According to \cite{technical_indicator}, ``Technical indicators are heuristic or pattern-based signals produced by the price, volume, 
and/or open interest of a security or contract used by traders who follow technical analysis.
By analyzing historical data, technical analysts use indicators to predict future price movements."

We are trying to use technical indicators to help decide when to entry or exit to make a profit trade. Techinal indicators
that we will use are RSI (Relative Strength Index) and Bollinger Bands.

\subsubsection*{RSI - Relative Strength Index}
According to \cite{rsi} ``The relative strength index (RSI) is a momentum indicator used in technical analysis. 
RSI measures the speed and magnitude of a security's recent price changes to evaluate overvalued or 
undervalued conditions in the price of that security."
RSI is a value in range $[0, 100]$ and the value at time $t$ is defined as follows:
$$
    \text{RSI} = 100 - \frac{100}{1 + \text{RS}}
$$
where $\text{RS}$ is the relative strength of the last $n$ sessions and its defined as:
$$
    \text{RS} = \frac{\text{AverageGain}_t(n)}{\text{AverageLoss}_t(n)}
$$
where $\text{AverageGain}_t(n)$ and $\text{AverageLoss}_t(n)$ are the average of the gains ($\text{price}_t - \text{price}_{t-1}$) 
and losses ($\text{price}_t < \text{price}_{t-1}$), obtained in the last n sessions. That is, from time $t - (n - 1)$ to time $t$. 
However, theses values are usually from estimated using the following smoothing equations:
$$
    \text{AverageGain}_t(n) = \frac{\text{AverageGain}_{t-1}(n) \cdot (n-1) + \text{gain}_t}{n}
$$ 
$$
    \text{AverageLoss}_t(n) = \frac{\text{AverageGain}_{t-1}(n) \cdot (n-1) + \text{loss}_t}{n}
$$ 
If a session $t$ result in gain then $\text{loss}_t = 0$ and, if results in loss the $\text{profit}_t = 0$. Commom number of sessions
are $n = 14$ and a common interpretation of the RSI index is that it suggests oversold at value $< 30$, and overbought for value $ > 70$ 

\subsubsection*{Bollinger Bands}
According to \cite{bb}, ``A Bollinger Band is a technical analysis tool defined by a set of trendlines plotted two standard deviations
(positively and negatively) away from a simple moving average (SMA) of a security's price.'' 
Bollinger Bands ($\text{BOLU}$ for upper band, $\text{BOLD}$ for lower band) at time $t$ are defined as:
$$
\text{BOLU} = \text{MA}(n) + m * \sigma(n)
$$
$$
\text{BOLD} = \text{MA}(n) - m * \sigma(n)
$$
where $m$ is number of standard deviations (usually 2), and both $\text{MA}(n)$ and $\sigma(n)$ of the last $n$ sessions are defined as:
$$
\text{MA}(n) = \frac{\sum_{i = 1}^{n}{p_i}}{n}
$$
$$
\sigma(n) = \sqrt{\frac{\sum_{i = 1}^{n}{p_i - \text{MA}(n)}}{n}}
$$
where each $p_i$ is a typical price calulated as $p_i = \frac{(\text{high}_i + \text{low}_i + \text{close}_i)}{3}$

Common number of sessions are $n = 20$ and a common interpretation is the closer the prices move to $\text{BOLU}$, 
the more overbought the market, and the closer the prices move to the lower band, the more oversold the market.

\section*{Entry rules}
Entry rule tell you that at a current time, should you open a position or not? and it is what we are going to make with fuzzy logic.
For clarifying, a position in this report will be in these 2 types:
\begin{enumerate}
    \item LONG, which we gain profit from price increasing.
    \item SHORT, which we gain profit from price decreasing.
\end{enumerate}

\subsection*{Classic rules}
The classic rules is based on set of fixed rules where the input are exact number and the outputs are binary values (1 - yes, 0 - no). 
The condition is from the trader's belief which should have some uncertainty, but the classic rules can't include that uncertainty in the rules.
An examples of classic rules can be seen on \cref*{table:1} and \cref*{table:2} which could be more complex and practical by adding more ``useful'' indicators.
\begin{table}[htp]
	\centering
	\begin{tabular}{l c c c c c}
		\toprule
        {} & {RSI} & {} & {Bollinger Bands} & {} & {LONG} \\ 
        \midrule
        If & $<30$ & \& & $|\text{price} - \text{BOLD}| < 10$ & then & 1 \\
        elseif & $<30$ & \& & $\text{price} - \text{BOLD} < -10$ & then & 1 \\
        else &  &  &  & then & 0 \\
        \bottomrule
    \end{tabular} 
    \caption{Examples for classic rule (LONG signal).}
	\label{table:1}
\end{table}
\begin{table}[htp]
	\centering
	\begin{tabular}{l c c c c c}
		\toprule
        {} & {RSI} & {} & {Bollinger Bands} & {} & {SHORT} \\ 
        \midrule
        If & $ >70$ & \& & $|\text{price} - \text{BOLU}| < 10$ & then & 1 \\
        elseif & $ >70$ & \& & $\text{price} - \text{BOLU} > 10$ & then & 1 \\
        else &  &  &  & then & 0 \\
        \bottomrule
    \end{tabular} 
    \caption{Examples for classic rule (SHORT signal).}
	\label{table:2}
\end{table}

\subsection*{Fuzzy rules}

\newpage
\printbibliography

\end{document}