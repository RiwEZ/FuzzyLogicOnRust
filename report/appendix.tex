\definecolor{bg}{rgb}{0.97,0.97,0.98}

\section*{Appendix}
\noindent main.rs
\begin{minted}[fontsize=\footnotesize, bgcolor=bg]{rust}
pub mod backtest;
pub mod data;
pub mod rule;
pub mod set;
pub mod shape;

use chrono::offset::{Local, TimeZone};
use chrono::Date;
use data::read_csv;
use plotters::prelude::*;
use rule::FuzzyEngine;
use set::{arange, LinguisticVar};
use shape::{trapezoidal, triangular};
use std::error::Error;

fn parse_time(t: &str) -> Date<Local> {
    Local
        .datetime_from_str(t, "%Y-%m-%d %H:%M:%S %Z")
        .unwrap()
        .date()
}

fn max_of_vec(vec: &Vec<f64>) -> f64 {
    vec.iter().fold(f64::NAN, |max, &val| val.max(max))
}

fn min_of_vec(vec: &Vec<f64>) -> f64 {
    vec.iter().fold(f64::NAN, |min, &val| val.min(min))
}

fn plot(date: &Vec<Date<Local>>, data: [&Vec<f64>; 3], path: &str) -> Result<(), Box<dyn Error>> {
    let root = SVGBackend::new(path, (1024, 1024)).into_drawing_area();
    root.fill(&WHITE)?;
    let area = root.split_evenly((3, 1));

    let colors = [&BLUE, &GREEN, &RED];
    let names = ["ETH Price Chart", "Long Signal", "Short Signal"];
    let limits = [
        (min_of_vec(&data[0]), max_of_vec(&data[0])),
        (0.0, 80.0),
        (0.0, 80.0),
    ];
    for (i, draw_area) in area.iter().enumerate() {
        let mut chart = ChartBuilder::on(&draw_area)
            .caption(names[i], ("Hack", 44, FontStyle::Bold).into_font())
            .set_label_area_size(LabelAreaPosition::Left, 70)
            .set_label_area_size(LabelAreaPosition::Bottom, 60)
            .margin_right(20)
            .build_cartesian_2d(date[0]..date[date.len() - 1], limits[i].0..limits[i].1)?;

        chart
            .configure_mesh()
            .y_max_light_lines(0)
            .x_labels(11)
            .x_label_formatter(&|v| format!("{}", v.format("%Y-%m-%d")))
            .x_label_style(("Hack", 16).into_font())
            .y_label_style(("Hack", 16).into_font())
            .draw()?;

        chart.draw_series(LineSeries::new(
            date.iter().zip(data[i].iter()).map(|(d, p)| (*d, *p)),
            colors[i].stroke_width(2),
        ))?;
    }
    root.present()?;
    Ok(())
}

fn main() -> Result<(), Box<dyn std::error::Error>> {
    let rsi_var = LinguisticVar::new(
        vec![
            (&triangular(0f64, 1.0, 30f64), "low"),
            (&triangular(50f64, 1.0, 30f64), "medium"),
            (&triangular(100f64, 1.0, 30f64), "high"),
        ],
        arange(0f64, 100f64, 0.01),
    );
    let bb_var = LinguisticVar::new(
        vec![
            (&triangular(-120f64, 1.0, 30f64), "long"),
            (&trapezoidal(-100f64, -50f64, 50f64, 100f64, 1.0), "wait"),
            (&triangular(120f64, 1.0, 30f64), "short"),
        ],
        arange(-150f64, 150f64, 0.01),
    );
    let long = LinguisticVar::new(
        vec![
            (&triangular(0f64, 1.0, 15f64), "weak"),
            (&triangular(30f64, 1.0, 30f64), "strong"),
            (&triangular(100f64, 1.0, 60f64), "verystrong"),
        ],
        arange(0f64, 100f64, 0.01),
    );
    let short = LinguisticVar::new(
        vec![
            (&triangular(0f64, 1.0, 15f64), "weak"),
            (&triangular(30f64, 1.0, 30f64), "strong"),
            (&triangular(100f64, 1.0, 60f64), "verystrong"),
        ],
        arange(0f64, 100f64, 0.01),
    );

    //rsi_var.plot("rsi".into(), "img/rsi.svg".into())?;
    //bb_var.plot("bollinger bands".into(), "img/bb.svg".into())?;
    //long.plot("long, short".into(), "img/ls.svg".into())?;

    let mut f_engine = FuzzyEngine::new([rsi_var, bb_var], [long, short]);

    f_engine.add_rule(["high", "long"], ["weak", "weak"]);
    f_engine.add_rule(["high", "wait"], ["weak", "strong"]);
    f_engine.add_rule(["high", "short"], ["weak", "verystrong"]);

    f_engine.add_rule(["medium", "long"], ["weak", "strong"]);
    f_engine.add_rule(["medium", "wait"], ["weak", "weak"]);
    f_engine.add_rule(["medium", "short"], ["strong", "weak"]);

    f_engine.add_rule(["low", "long"], ["verystrong", "weak"]);
    f_engine.add_rule(["low", "wait"], ["strong", "weak"]);
    f_engine.add_rule(["low", "short"], ["weak", "weak"]);

    let data = read_csv("eth.csv");
    let rsi = data::rsi(&data, 14)[2256..].to_vec();
    let bb = data::bb(&data, 20)[2256..].to_vec();
    let price: Vec<f64> = data[2256..].iter().map(|x| x.price).collect();
    let bb_inputs: Vec<f64> = price
        .iter()
        .zip(bb.iter())
        .map(|(p, y)| 100.0 * (p - y.0) / (2.0 * y.1))
        .collect();

    let date: Vec<Date<Local>> = data[2256..]
        .iter()
        .map(|x| parse_time(x.snapped_at.as_str()))
        .collect();

    let mut long_singal: Vec<f64> = vec![];
    let mut short_singal: Vec<f64> = vec![];
    for i in 0..price.len() {
        let result = f_engine.calculate([rsi[i], bb_inputs[i]]);
        long_singal.push(result[0].centroid_defuzz());
        short_singal.push(result[1].centroid_defuzz());
    }
    /*
    plot(
        &date,
        [&price, &long_singal, &short_singal],
        "img/chart.svg",
    )?;
    */

    backtest::f_backtest(&price, &long_singal, false);
    backtest::f_backtest(&price, &short_singal, true);
    backtest::c_backtest(&price, &rsi, &bb, false);
    backtest::c_backtest(&price, &rsi, &bb, true);

    Ok(())
}
\end{minted}
\noindent rule.rs
\begin{minted}[fontsize=\footnotesize, bgcolor=bg]{rust}
use crate::set::*;

pub struct FuzzyEngine<const N: usize, const M: usize> {
    inputs_var: [LinguisticVar; N],
    outputs_var: [LinguisticVar; M],
    rules: Vec<(Vec<String>, Vec<String>)>, // list of ([input1_term, input2_term, ...] -> output_term)
}

impl<const N: usize, const M: usize> FuzzyEngine<N, M> {
    pub fn new(
        inputs_var: [LinguisticVar; N],
        output_var: [LinguisticVar; M],
    ) -> FuzzyEngine<N, M> {
        FuzzyEngine {
            inputs_var,
            outputs_var: output_var,
            rules: Vec::<(Vec<String>, Vec<String>)>::new(),
        }
    }

    pub fn add_rule(&mut self, cond: [&str; N], res: [&str; M]) {
        for i in 0..self.inputs_var.len() {
            self.inputs_var[i].term(&cond[i]); // check if term "cond[i]" exist
        }
        for i in 0..self.outputs_var.len() {
            self.outputs_var[i].term(&res[i]); // term() check if term "res" is exist
        }

        let conditions: Vec<String> = cond.iter().map(|x| x.to_string()).collect();
        let results: Vec<String> = res.iter().map(|x| x.to_string()).collect();
        self.rules.push((conditions, results));
    }

    pub fn calculate(&self, inputs: [f64; N]) -> Vec<FuzzySet> {
        let mut temp: Vec<Vec<FuzzySet>> = vec![];
        for j in 0..self.rules.len() {
            let mut aj = f64::MAX;
            for i in 0..self.rules[j].0.len() {
                let fuzzy_set = self.inputs_var[i].term(&self.rules[j].0[i]);
                let v = fuzzy_set.degree_of(inputs[i]);
                aj = aj.min(v);
            }

            let mut t: Vec<FuzzySet> = vec![];
            for i in 0..self.rules[j].1.len() {
                t.push(
                    self.outputs_var[i]
                        .term(&self.rules[j].1[i])
                        .min(aj, format!("f{}", j)),
                );
            }

            temp.push(t);
        }
        let mut res: Vec<FuzzySet> = vec![];
        for i in 0..M {
            res.push(temp[0][i].std_union(&temp[0][i], "".into()));
        }
        for j in 1..temp.len() {
            for i in 0..M {
                res[i] = res[i].std_union(&temp[j][i], "".into());
            }
        }
        res
    }
}

#[cfg(test)]
mod tests {
    use super::*;
    use crate::shape::*;
    use std::error::Error;

    #[test]
    #[should_panic]
    fn test_adding_rule() {
        let rsi = LinguisticVar::new(
            vec![
                (&triangular(20f64, 1.0, 20f64), "low"),
                (&triangular(80f64, 1.0, 20f64), "high"),
            ],
            arange(0f64, 100f64, 0.01),
        );

        let mut f_engine = FuzzyEngine::new([rsi.clone()], [rsi]);
        f_engine.add_rule(["medium".into()], ["low".into()]);
    }

    #[test]
    fn basic_test() -> Result<(), Box<dyn Error>> {
        let rsi = LinguisticVar::new(
            vec![
                (&triangular(20f64, 1.0, 20f64), "low"),
                (&triangular(80f64, 1.0, 20f64), "high"),
            ],
            arange(0f64, 100f64, 0.01),
        );

        let ma = LinguisticVar::new(
            vec![(&triangular(30f64, 0.8, 20f64), "low")],
            arange(0f64, 100f64, 0.01),
        );

        let trend = LinguisticVar::new(
            vec![(&triangular(30f64, 1f64, 30f64), "weak")],
            arange(0f64, 100f64, 0.01),
        );

        rsi.plot("rsi".into(), "img/test/rsi.svg".into())?;
        ma.plot("ma".into(), "img/test/ma.svg".into()).unwrap();
        ma.plot("ma".into(), "img/test/ma.svg".into()).unwrap();
        trend
            .plot("trend".into(), "img/test/trend.svg".into())
            .unwrap();

        let mut f_engine = FuzzyEngine::new([rsi, ma], [trend]);

        f_engine.add_rule(["low", "low"], ["weak"]);
        f_engine.add_rule(["high", "low"], ["weak"]);

        let res = f_engine.calculate([15f64, 25f64]);
        res[0].plot("t2".into(), "img/test/t2.svg".into())?;

        Ok(())
    }
}
\end{minted}
\noindent set.rs
\begin{minted}[fontsize=\footnotesize, bgcolor=bg]{rust}
use crate::shape::*;
use plotters::prelude::*;
use std::error::Error;

pub fn arange(start: f64, stop: f64, interval: f64) -> Vec<f64> {
    if stop < start {
        panic!("end can not be less than start");
    } else if interval <= 0f64 {
        panic!("interval must be > 0");
    }

    let mut members: Vec<f64> = vec![];
    let r = 1.0 / interval;
    let mut n = start;
    while n <= stop {
        members.push(n);
        n += interval;
        if interval < 1.0 {
            n = (n * r).round() / r;
        }
    }
    members
}

#[derive(Clone)]
pub struct LinguisticVar {
    pub sets: Vec<FuzzySet>,
    pub universe: Vec<f64>,
}

impl LinguisticVar {
    pub fn new(inputs: Vec<(&dyn Shape, &str)>, universe: Vec<f64>) -> LinguisticVar {
        let mut sets: Vec<FuzzySet> = vec![];
        for item in inputs {
            sets.push(FuzzySet::new(&universe, item.0, item.1.to_string()));
        }
        LinguisticVar { sets, universe }
    }

    pub fn term(&self, name: &str) -> &FuzzySet {
        match self.sets.iter().find(|x| x.name == name.to_string()) {
            Some(x) => x,
            None => panic![
                "there're no fuzzy set name {} in this linguistic variable",
                name
            ],
        }
    }

    pub fn plot(&self, name: String, path: String) -> Result<(), Box<dyn Error>> {
        let root = SVGBackend::new(&path, (1024, 768)).into_drawing_area();
        root.fill(&WHITE)?;

        let mut chart = ChartBuilder::on(&root)
            .caption(name, ("Hack", 70, FontStyle::Bold).into_font())
            .set_label_area_size(LabelAreaPosition::Left, 60)
            .set_label_area_size(LabelAreaPosition::Bottom, 60)
            .margin(60)
            .build_cartesian_2d(
                self.universe[0]..self.universe[self.universe.len() - 1],
                0f64..1f64,
            )?;

        chart
            .configure_mesh()
            .disable_x_mesh()
            .y_max_light_lines(0)
            .x_labels(5)
            .y_labels(5)
            .x_label_style(("Hack", 40).into_font())
            .y_label_style(("Hack", 40).into_font())
            .draw()?;

        for i in 0..self.sets.len() {
            let color = Palette99::pick(i);
            chart
                .draw_series(LineSeries::new(
                    self.universe
                        .iter()
                        .zip(self.sets[i].membership.iter())
                        .map(|(x, y)| (*x, *y)),
                    color.mix(0.5).stroke_width(4),
                ))?
                .label(self.sets[i].name.clone())
                .legend(move |(x, y)| {
                    PathElement::new([(x, y), (x + 20, y)], color.filled().stroke_width(4))
                });
            // there're a problem with styling legend
        }

        chart
            .configure_series_labels()
            .label_font(("Hack", 50).into_font())
            .background_style(&WHITE)
            .border_style(&BLACK)
            .draw()?;

        root.present()?;
        Ok(())
    }
}

#[derive(Debug, Clone)]
pub struct FuzzySet {
    pub name: String,
    pub universe: Vec<f64>, // universe of discourse that own this set
    pub membership: Vec<f64>,
}

impl FuzzySet {
    pub fn new(universe: &Vec<f64>, fuzzy_f: &dyn Shape, name: String) -> FuzzySet {
        let mut membership: Vec<f64> = vec![];
        for i in 0..universe.len() {
            membership.push(fuzzy_f.function(universe[i]));
        }
        FuzzySet {
            name: name.to_string(),
            universe: universe.clone(),
            membership,
        }
    }

    pub fn plot(&self, name: String, path: String) -> Result<(), Box<dyn Error>> {
        let root = SVGBackend::new(&path, (1024, 768)).into_drawing_area();
        root.fill(&WHITE)?;

        let mut chart = ChartBuilder::on(&root)
            .caption(name, ("Hack", 44, FontStyle::Bold).into_font())
            .set_label_area_size(LabelAreaPosition::Left, 60)
            .set_label_area_size(LabelAreaPosition::Bottom, 60)
            .margin(20)
            .build_cartesian_2d(
                self.universe[0]..self.universe[self.universe.len() - 1],
                0f64..1f64,
            )?;

        chart
            .configure_mesh()
            .disable_x_mesh()
            .disable_y_mesh()
            .draw()?;

        let color = Palette99::pick(0);
        chart
            .draw_series(LineSeries::new(
                self.universe
                    .iter()
                    .zip(self.membership.iter())
                    .map(|(x, y)| (*x, *y)),
                color.stroke_width(2),
            ))?
            .label(self.name.clone())
            .legend(move |(x, y)| PathElement::new([(x, y), (x + 20, y)], color.filled()));

        chart
            .configure_series_labels()
            .label_font(("Hack", 14).into_font())
            .background_style(&WHITE)
            .border_style(&BLACK)
            .draw()?;

        root.present()?;

        Ok(())
    }

    pub fn degree_of(&self, input: f64) -> f64 {
        // edge case
        if input < self.universe[0] {
            return self.membership[0];
        } else if input > self.universe[self.universe.len() - 1] {
            return self.membership[self.membership.len() - 1];
        }
        let mut min_x = f64::MAX;
        let mut j: usize = 0;
        for (i, x) in self.universe.iter().enumerate() {
            let diff = (x - input).abs();
            if diff < min_x {
                j = i;
                min_x = diff;
            }
        }
        self.membership[j]
    }

    pub fn centroid_defuzz(&self) -> f64 {
        let top_sum = self
            .universe
            .iter()
            .enumerate()
            .fold(0.0, |s, (x, y)| s + (self.membership[x] * y));
        let bot_sum = self.membership.iter().fold(0.0, |s, v| s + v);
        if bot_sum == 0.0 {
            return 0.0;
        }
        top_sum / bot_sum
    }

    pub fn min(&self, input: f64, name: String) -> FuzzySet {
        let mut membership: Vec<f64> = vec![];
        for i in 0..self.membership.len() {
            membership.push(self.membership[i].min(input));
        }
        FuzzySet {
            name: name.to_string(),
            universe: self.universe.clone(),
            membership,
        }
    }

    pub fn std_union(&self, set: &FuzzySet, name: String) -> FuzzySet {
        // check if domain is equal or not?
        if self.universe != set.universe {
            panic!("domain needs to be equal");
        }

        // if equal
        let mut membership: Vec<f64> = vec![];
        for i in 0..self.membership.len() {
            membership.push(self.membership[i].max(set.membership[i]));
        }
        FuzzySet {
            name: name.to_string(),
            universe: self.universe.clone(),
            membership,
        }
    }

    pub fn std_intersect(&self, set: &FuzzySet, name: String) -> FuzzySet {
        // check if domain is equal or not?
        if self.universe != set.universe {
            panic!("domain needs to be equal");
        }

        // if equal
        let mut membership: Vec<f64> = vec![];
        for i in 0..self.membership.len() {
            membership.push(self.membership[i].min(set.membership[i]));
        }
        FuzzySet {
            name: name.to_string(),
            universe: self.universe.clone(),
            membership,
        }
    }
}

#[cfg(test)]
mod tests {
    use super::*;

    #[test]
    fn test_arange() {
        assert_eq!(arange(0f64, 5f64, 1f64), vec![0.0, 1.0, 2.0, 3.0, 4.0, 5.0]);
        assert_eq!(
            arange(0f64, 0.5f64, 0.1f64),
            vec![0.0, 0.1, 0.2, 0.3, 0.4, 0.5]
        );
    }

    #[test]
    fn test_degree() {
        let s1 = FuzzySet::new(
            &arange(0.0, 10.0, 0.01),
            &triangular(5f64, 0.8f64, 3f64),
            "f1".into(),
        );

        assert_eq!(s1.degree_of(11.0f64), 0.0);
        assert_eq!(s1.degree_of(5.0f64), 0.8);
        assert_eq!(s1.degree_of(3.5f64), 0.4);
        assert_eq!(s1.degree_of(0.0f64), 0.0);
        assert_eq!(s1.degree_of(-1.0f64), 0.0);
    }

    #[test]
    fn linguistic() {
        let var1 = LinguisticVar::new(
            vec![
                (&triangular(5f64, 0.8, 3f64), "normal"),
                (&triangular(3f64, 0.8, 1.5f64), "weak"),
            ],
            arange(0f64, 10f64, 0.01),
        );

        assert_eq!(var1.term("normal").degree_of(5.0), 0.8);
        assert_eq!(var1.term("weak").degree_of(3.0), 0.8);

        var1.plot("var1".into(), "img/t.svg".into()).unwrap();
    }
}

\end{minted}
\noindent shape.rs
\begin{minted}[fontsize=\footnotesize, bgcolor=bg]{rust}
pub trait Shape {
    fn function(&self, x: f64) -> f64;
}

pub struct Triangular {
    a: f64,
    b: f64,
    s: f64,
}

impl Shape for Triangular {
    fn function(&self, x: f64) -> f64 {
        if (self.a - self.s) <= x && x <= (self.a + self.s) {
            return self.b * (1.0 - (x - self.a).abs() / self.s);
        }
        0.0
    }
}

pub fn triangular(a: f64, b: f64, s: f64) -> Triangular {
    Triangular { a, b, s }
}

pub struct Trapezoidal {
    a: f64,
    b: f64,
    c: f64,
    d: f64,
    e: f64,
}

impl Shape for Trapezoidal {
    fn function(&self, x: f64) -> f64 {
        if x >= self.a && x < self.b {
            return ((x - self.a) * self.e) / (self.b - self.a);
        } else if x >= self.b && x <= self.c {
            return self.e;
        } else if x > self.c && x <= self.d {
            return self.e * (1.0 - (x - self.c).abs() / (self.d - self.c));
        }
        0.0
    }
}

pub fn trapezoidal(a: f64, b: f64, c: f64, d: f64, e: f64) -> Trapezoidal {
    Trapezoidal { a, b, c, d, e }
}

\end{minted}
\noindent data.rs
\begin{minted}[fontsize=\footnotesize, bgcolor=bg]{rust}
use serde::Deserialize;
use std::fs;

#[derive(Deserialize, Debug)]
pub struct Record {
    pub snapped_at: String,
    pub price: f64,
}

/// return (gain, loss)
fn gain_loss(p0: f64, p1: f64) -> (f64, f64) {
    let mut res = (0.0, 0.0);
    if p1 > p0 {
        res.0 = p1 - p0
    } else {
        res.1 = -(p1 - p0)
    }
    res
}

pub fn rsi(data: &Vec<Record>, n: usize) -> Vec<f64> {
    let mut rsi: Vec<f64> = vec![];

    let mut gains: Vec<f64> = vec![];
    let mut losses: Vec<f64> = vec![];

    let mut avg_g: Vec<f64> = vec![];
    let mut avg_l: Vec<f64> = vec![];

    for (i, item) in data.iter().enumerate() {
        if i <= n + 1 && i > 0 {
            let gl = gain_loss(data[i - 1].price, item.price);
            gains.push(gl.0);
            losses.push(gl.1);
        }

        if i <= n {
            rsi.push(-1.0);
            avg_g.push(0.0);
            avg_l.push(0.0);
        } else if i == n + 1 {
            let avg_gain = gains.iter().sum::<f64>() / (n as f64);
            let avg_loss = losses.iter().sum::<f64>() / (n as f64);
            rsi.push(100f64 - 100f64 / (1.0 + (avg_gain / avg_loss)));
            avg_g.push(avg_gain);
            avg_l.push(avg_loss);
        } else {
            let gl = gain_loss(data[i - 1].price, item.price);
            let avg_gain = (avg_g[i - 1] * (n - 1) as f64 + gl.0) / (n as f64);
            let avg_loss = (avg_l[i - 1] * (n - 1) as f64 + gl.1) / (n as f64);
            rsi.push(100f64 - 100f64 / (1.0 + (avg_gain / avg_loss)));
            avg_g.push(avg_gain);
            avg_l.push(avg_loss);
        }
    }
    rsi
}

/// return vector of (ma, std)
pub fn bb(data: &Vec<Record>, n: usize) -> Vec<(f64, f64)> {
    let mut bb: Vec<(f64, f64)> = vec![];
    for (i, _) in data.iter().enumerate() {
        if i < n {
            bb.push((-1.0, -1.0));
            continue;
        }

        let mut sum = 0.0;
        for j in (i - n)..i {
            sum += data[j].price;
        }
        let ma = sum / (n as f64);

        let mut std_sum = 0.0;
        for j in (i - n)..i {
            std_sum += (data[j].price - ma).powi(2);
        }
        let std = (std_sum / n as f64).sqrt();
        bb.push((ma, std));
    }
    bb
}

/// read coingecko csv data
pub fn read_csv(path: &str) -> Vec<Record> {
    let contents = fs::read_to_string(path).unwrap();

    let mut rdr = csv::Reader::from_reader(contents.as_bytes());
    let mut data: Vec<Record> = vec![];
    for result in rdr.deserialize() {
        let record: Record = result.unwrap();
        data.push(record);
    }
    data
}

\end{minted}
\noindent backtest.rs
\begin{minted}[fontsize=\footnotesize, bgcolor=bg]{rust}  
// get long, short signal as an input
// iterate over all data with inital capital 1000$
// entry when signal is >= 40
// entry size is 20% of capital
// stop-loss when price go down 10%
// take profit when price go up 20%

struct Position {
    at_price: f64,
    at_time: usize,
    amount: f64,
    realize: bool,
}

impl Position {
    pub fn new(price: f64, money: f64, time: usize) -> Position {
        Position {
            at_price: price,
            at_time: time,
            amount: money / price,
            realize: false,
        }
    }
}

fn realizing_pos(capital: f64, price: &Vec<f64>, pos_list: &mut Vec<Position>, pos_type: bool) {
    let mut profit: Vec<f64> = vec![];
    let mut losses: Vec<f64> = vec![];
    for (i, p) in price.iter().enumerate() {
        for j in 0..pos_list.len() {
            if i <= pos_list[j].at_time || pos_list[j].realize == true {
                continue;
            }
            let diff = (p - pos_list[j].at_price) / pos_list[j].at_price * 100.0;
            if !pos_type {
                if diff > 20.0 {
                    profit.push((p - pos_list[j].at_price) * pos_list[j].amount);
                    pos_list[j].realize = true;
                } else if diff < -10.0 {
                    losses.push((p - pos_list[j].at_price) * pos_list[j].amount);
                    pos_list[j].realize = true;
                }
            } else {
                if diff < -20.0 {
                    profit.push(-1.0 * (p - pos_list[j].at_price) * pos_list[j].amount);
                    pos_list[j].realize = true;
                } else if diff > 10.0 {
                    losses.push(-1.0 * (p - pos_list[j].at_price) * pos_list[j].amount);
                    pos_list[j].realize = true;
                }
            }
        }
    }

    let total_profit = profit.iter().fold(0.0, |s, x| s + x);
    let total_losses = losses.iter().fold(0.0, |s, x| s + x);
    println!("total trade: {:.3}", pos_list.len());
    println!("net profit: {:.3}", total_profit + total_losses);
    println!("count: {}, profits: {:.3}", profit.len(), total_profit);
    println!("count: {}, losses: {:.3}", losses.len(), total_losses);
    println!("---------------");
    /*
    println!(
        "total capital(+net_profit)): {}",
        capital + (1000.0 - capital) + total_profit + total_losses
    );
    */
}

/// Fuzzy BackTest
/// pos_type - false for long, true for short
pub fn f_backtest(price: &Vec<f64>, signal: &Vec<f64>, pos_type: bool) {
    let mut capital = 1000.0;
    let mut pos_list: Vec<Position> = vec![];

    for (i, p) in price.iter().enumerate() {
        if signal[i] >= 40.0 {
            if capital > 0.0 {
                let pos = Position::new(*p, 100.0, i);
                capital -= 100.0;
                pos_list.push(pos);
            }
        }
    }

    realizing_pos(capital, price, &mut pos_list, pos_type)
}

pub fn c_backtest(price: &Vec<f64>, rsi: &Vec<f64>, bb: &Vec<(f64, f64)>, pos_type: bool) {
    let mut capital = 1000.0;
    let mut pos_list: Vec<Position> = vec![];

    for (i, p) in price.iter().enumerate() {
        let beta = (p - bb[i].0) / (2.0 * bb[i].1);
        if !pos_type {
            if rsi[i] < 30.0 && beta < -0.9 {
                if capital > 0.0 {
                    let pos = Position::new(*p, 100.0, i);
                    capital -= 100.0;
                    pos_list.push(pos);
                } else if rsi[i] < 30.0 && beta >= -0.9 && beta < 0.0 {
                    let pos = Position::new(*p, 100.0, i);
                    capital -= 100.0;
                    pos_list.push(pos);
                }
            }
        } else {
            if rsi[i] > 70.0 && beta < 0.9 {
                if capital > 0.0 {
                    let pos = Position::new(*p, 100.0, i);
                    capital -= 100.0;
                    pos_list.push(pos);
                } else if rsi[i] > 70.0 && beta <= 0.9 && beta > 0.0 {
                    let pos = Position::new(*p, 100.0, i);
                    capital -= 100.0;
                    pos_list.push(pos);
                }
            }
        }
    }

    realizing_pos(capital, price, &mut pos_list, pos_type)
}
\end{minted}